\documentclass[11pt]{article}


%%%%%%%%%%%%%%%%%%%%%%%%%%%%%%%%
%%%%%%%%%%%%%%%%%%%%%%%%%%%%%%%%
%% PACKAGES %%%%%%%%%%%%%%%%%%%%
%%%%%%%%%%%%%%%%%%%%%%%%%%%%%%%%
%%%%%%%%%%%%%%%%%%%%%%%%%%%%%%%%

\usepackage[margin=0.9in, top=0.8in, bottom=1.0in]{geometry}
\usepackage[charter]{mathdesign} % Main font.
\usepackage{qpxmath} % Palatino-based math font, better than charter.
\usepackage[scaled]{beramono} % Lovely monospace font
\usepackage[T1]{fontenc}
\usepackage{amsmath}
\usepackage{mathtools}
\usepackage{mathdots}
\usepackage{graphicx}
\usepackage[update,prepend]{epstopdf} % To use eps files.
\usepackage{microtype}
\usepackage{titlesec} % Custom section headings.
\usepackage{xcolor}
\usepackage{xspace}
\usepackage{xfrac}
\usepackage{calc}
\usepackage{listings} % Code listings.
\usepackage{matlab-prettifier} % MATLAB code listings
\usepackage[labelfont=bf]{caption} % Figure captions.
\usepackage{enumitem} % Fine tuning enumerations.
\usepackage{floatrow} % Captions to the right of figures.
\usepackage[numbers]{natbib}
\usepackage{wrapfig}

% Packages to makes tables pretty.
\usepackage{array}
\usepackage{booktabs}
\setlength{\heavyrulewidth}{1.5pt}
\setlength{\abovetopsep}{4pt}

% Fancyhdr package stuff...
\usepackage{fancyhdr}
\setlength{\headheight}{0pt}
\setlength{\footskip}{50pt}
\renewcommand{\headrulewidth}{0pt}
\renewcommand{\footrulewidth}{0pt}

%%%%%%%%%%%%%%%%%%%%%%%%%%%%%%%%
%%%%%%%%%%%%%%%%%%%%%%%%%%%%%%%%
%% SETTINGS %%%%%%%%%%%%%%%%%%%%
%%%%%%%%%%%%%%%%%%%%%%%%%%%%%%%%
%%%%%%%%%%%%%%%%%%%%%%%%%%%%%%%%

% Path to look for graphics
\graphicspath{{../images/}}
%\epstopdfsetup{outdir=../images/}

% Caption spacing
\setlength{\abovecaptionskip}{0pt}

% List spacing
\setlist{noitemsep}

% Math operator font
\DeclareSymbolFont{sfoperators}{OT1}{cmss}{m}{n}
\DeclareSymbolFontAlphabet{\mathsf}{sfoperators}
\makeatletter
\def\operator@font{\mathgroup\symsfoperators}
\makeatother

%% No indent all paragraphs
%\setlength{\parindent}{0in}

% Figure references
\newcommand{\figref}[1]{Figure~\ref{#1}}

% Special format section headings
\titleformat{\section}%
	{\large\bf\scshape}% Text formatting
	{\arabic{section}}% Number
	{1em}% Space between number and text
	{}% Code before
	[]% Code after
\titleformat{\subsection}%
	{\normalsize\bf\scshape}% Text formatting
	{\arabic{section}.\arabic{subsection}}% Number
	{1em}% Space between number and text
	{}% Code before
	[]% Code after
%\titleformat{\subsubsection}%
%	{\color{blue}}% Text formatting
%	{\arabic{subsubsection} $\rightarrow$}% Number
%	{1em}% Space between number and text
%	{}% Code before
%	[]% Code after

\definecolor{mygray}{rgb}{0.4, 0.4, 0.4}
\lstset{
style=Matlab-editor,
mlscaleinline=false,
basicstyle=\ttfamily\lst@ifdisplaystyle\scriptsize\fi,
frame=single,
rulecolor=\color{mygray},
numbers=left,
numbersep=10pt,
numberstyle=\footnotesize \ttfamily \color{mygray},
xleftmargin=30pt,
xrightmargin=5pt,
framexleftmargin=4pt,
framextopmargin=2pt
}

% Allow white-space to be eaten within any lst environments between returns.
\lstset{breaklines,breakatwhitespace}

%%%%%%%%%%%%%%%%%%%%%%%%%%%%%%%%
%%%%%%%%%%%%%%%%%%%%%%%%%%%%%%%%
%% COMMANDS %%%%%%%%%%%%%%%%%%%%
%%%%%%%%%%%%%%%%%%%%%%%%%%%%%%%%
%%%%%%%%%%%%%%%%%%%%%%%%%%%%%%%%

% Automated file inclusion for code listings
\makeatletter
\def\includecode{\@ifnextchar[{\@with}{\@without}}
\def\@with[#1]#2{
}
\def\@without#1{
  \lstinputlisting[caption=\ttfamily\protect\detokenize{#1}, escapechar=, frame=single]{../matlab_code/#1}
}
\makeatother

% Inline listing shorthand
\newcommand{\li}[1]{{\color{cyan!60!black}\sffamily\small{\detokenize{#1}}}}

% Wide tilde.
\newcommand{\wt}[1]{\ensuremath{\widetilde{#1}}}

% Degree symbol.
\newcommand{\degree}{\ensuremath{^\circ}}

% Partial derivatives
\newcommand{\pp}[2]{\ensuremath{\frac{\partial#1}{\partial#2}}}
\newcommand{\dd}[2]{\ensuremath{\frac{d#1}{d#2}}}
\newcommand{\DD}[2]{\ensuremath{\frac{D#1}{D#2}}}

% Superscript text: 1st, 2nd, 3rd, 4th
\newcommand{\suptext}[1]{\ensuremath{^\text{#1}}\xspace}
\newcommand{\st}{\suptext{st}}
\newcommand{\nd}{\suptext{nd}}
\newcommand{\rd}{\suptext{rd}}
\let\oldth\th % Reassign the current \th command
\renewcommand{\th}{\suptext{th}}

% Error function
\DeclareMathOperator\erf{erf}

% Big O notation
\newcommand{\bigo}{\ensuremath{\mathcal{O}}}

% Text max and min
\newcommand{\tmax}{\ensuremath{\text{max}}}
\newcommand{\tmin}{\ensuremath{\text{min}}}

% Norm
\newcommand{\norm}[1]{\ensuremath{\left| #1 \right|}}

% Bold vectors
% Option 1: Works on more than single tokens, but makes regular letters italic as well as bold.
%\renewcommand{\vec}[1]{\mathbold{#1}}
% Option 2: Only works if a single token is passed to the command, but makes regular letters bold only.
\newcommand{\mb}[1]{
	\ifcat\noexpand#1\relax
		\expandafter\mathbold
	\else
		\expandafter\mathbf
	\fi{{#1}}
}

% Underlines for tensor notation.
\newcommand{\tsr}[1]{\ensuremath{\underline{#1}}}
\newcommand{\tsrr}[1]{\ensuremath{\underline{\underline{#1}}}}


%%
%% DOCUMENT START
%%

\begin{document}

\pagestyle{fancyplain}
\lhead{}
\chead{}
\rhead{}
\lfoot{\hrule ASEN 6519}
\cfoot{\hrule \thepage}
\rfoot{\hrule Ryan Skinner}

\noindent
{\Large Final Project}
\hfill
{\large Ryan Skinner}
\\[0.5ex]
{\large ASEN 6519: Turbulence Modelling}
\hfill
{\large Due 2015/12/15}\\
\hrule
\vspace{6pt}

\vspace{0.5in}
\begin{center}
\LARGE Implementation and Validation of the Spalart-Allmaras \\ Curvature Correction in PHASTA
\end{center}
\vspace{0.2in}

%%%%%%%%%%%%%%%%%%%%%%%%%%%%%%%%%%%%%%%%%%%%%%%%%
%%%%%%%%%%%%%%%%%%%%%%%%%%%%%%%%%%%%%%%%%%%%%%%%%
\section{Introduction} %%%%%%%%%%%%%%%%%%%%%%%%%%
%%%%%%%%%%%%%%%%%%%%%%%%%%%%%%%%%%%%%%%%%%%%%%%%%
%%%%%%%%%%%%%%%%%%%%%%%%%%%%%%%%%%%%%%%%%%%%%%%%%

It is well-known that the presence of rotation and streamline curvature (RC) substantially alters the physics of turbulent shear flows. \citet{bradshaw1973} notes that these changes are ``surprisingly large,'' in that they are usually ``an order of magnitude more important than normal pressure gradients and other explicit terms'' in the RANS equations for curved flows. This can lead to significant effects on shear stresses and other quantities when the stream-wise radius of curvature is as large as one hundred times the shear layer thickness \citep{bradshaw1973}. For aerodynamicists in particular, RC phenomena have high impact on boundary layer development, turbulent mixing, and heat transfer in applications ranging from flow over high-camber airfoils to rapidly-rotating turbomachinery blades.

For computational studies to effectively guide developments in design areas dominated by RC-effects, it is imperative that the turbulence models employed capture these effects in some way. Reynolds stress transport (RST) models are commonly held superior to simpler eddy-viscosity models, because RC-terms appear explicitly in the Reynolds transport equation. Despite their accuracy, full RST models are much more costly, and adding an RC-correction term to the latter class of models would be a boon for workflows that require rapid design iteration.

The Spalart-Allmaras (SA) one equation turbulence model captures important features of aerodynamic flows involving complex geometry and adverse pressure gradients well, and is thus one of the most appropriate eddy-viscosity models for such studies \citep{spalart1992}. However, the original model neglects the effects of streamline curvature and rotation. To remedy this, \citet{shur2000} develop an RC-correction term that scales eddy viscosity production, giving rise to the SARC model. They validate their correction against experimental and DNS data of a number of canonical wall-bounded turbulent shear flows: 
\begin{itemize}
\item one-dimensional, fully-developed flow in a plane rotating channel,
\item one-dimensional, fully-developed flow in a curved channel,
\item two-dimensional flow in a channel with a U-turn, and
\item three-dimensional flow in a channel of rectangular cross-section with a 90\degree streamwise bend.
\end{itemize}
In all cases, the authors demonstrate substantial improvements of the SARC model over the standard SA model and in most cases the Menter two-equation shear stress transport (M-SST). These conclusions are based on predictions of mean velocity and wall shear stress distributions.

In the present project, the \citet{shur2000} curvature-correction (sans rotation terms) is added to the existing SA model in PHASTA, the Parallel Hierarchic Adaptive Stabilized Transient Analysis CFD code developed and maintained by Prof.\ Kenneth E.\ Jansen's group at the University of Colorado at Boulder. Our group focuses on aerodynamic flow control, which in many cases simplifies to ``the pursuit of bent streamlines''. Of particular interest to the author is the increased accuracy curvature-correction could bring to performance predictions of flow control strategies in an aggressive subsonic diffuser. Other applications include unsteady separation in flows over high-lift wing and tail configurations. Thus, the ability to run curvature-corrected RANS simulations will be a welcome addition to our research capabilities. To test our implementation's correctness, we simulate the 90\degree-bend case listed above, verify it using both the SA and SARC data of \citet{shur2000}, and validate it against the experimental data of \citet{kim1994}.

With the motivation clear, the remainder of this paper discusses the philosophy underpinning the mathematics of the RC-correction; the changes made to PHASTA during implementation; and the validation procedures employed, including geometry construction, meshing, and comparison to the published data mentioned above.

%%%%%%%%%%%%%%%%%%%%%%%%%%%%%%%%%%%%%%%%%%%%%%%%%
%%%%%%%%%%%%%%%%%%%%%%%%%%%%%%%%%%%%%%%%%%%%%%%%%
\section{Model Equations} %%%%%%%%%%%%%%%%%%%%%%%
%%%%%%%%%%%%%%%%%%%%%%%%%%%%%%%%%%%%%%%%%%%%%%%%%
%%%%%%%%%%%%%%%%%%%%%%%%%%%%%%%%%%%%%%%%%%%%%%%%%

\subsection{Spalart-Allmaras}

As the standard Spalart-Allmaras (SA) one-equation model \cite{spalart1992} is our point of departure for the RC-correction, a brief overview is apropos.

The SA model was developed as a middle-ground between algebraic and two-equation models. It sought to address algebraic models' shortcomings in massively-separated flows, while retaining some advantages of two-equation models and forgoing their additional computational complexity. It is tuned specifically for aerodynamic flows, which can exhibit substantial separation and involve complex geometries. Its derivation starts from a blank slate; production, transport, and diffusion terms are constructed from scratch using dimensional and invariance arguments applied to four canonical flows.

Fundamentally, the SA model solves a transport equation for the pseudo-eddy viscosity $\tilde{\nu}$, which is calibrated to behave properly within the log layer, and then scales it to the canonical eddy viscosity $\nu_T$ in a manner consistent with the viscous sublayer. PHASTA's current version of the SA model omits the original reference's trip term\footnote{Technically, PHASTA's version of the SA model corresponds to SA-noft2 on the NASA Turbulence Modelling Resource website.}, and chooses the vorticity magnitude as the scalar norm of the deformation tensor. The model, as implemented, can be written in full detail as
\begin{align}
	\phantom{W_{ij}}
	&\begin{aligned}
		\mathllap{\pp{\tilde{\nu}}{t}}
		&+ u_j \pp{\tilde{\nu}}{x_j}
		=
		c_{b1} \tilde{\Omega} \tilde{\nu}
		-
		c_{w1} f_w \left( \frac{\tilde{\nu}}{d} \right)^2
		+
		\frac{1}{\sigma} 
		\left[
		\pp{}{x_j}
			\left( (\nu + \tilde{\nu}) \pp{\tilde{\nu}}{x_j} \right)
			+ c_{b2} \pp{\tilde{\nu}}{x_j} \pp{\tilde{\nu}}{x_j}
		\right]
		\;,
		\label{eq:SA_transport}
	\end{aligned} \\[0.5cm]
	&\begin{aligned}
		\mathllap{\nu_T} &= \tilde{\nu} f_{v1}
		&\qquad
		f_{v1} &= \frac{\chi^3}{\chi^3 + c_{v1}^3}
		&\qquad
		\chi &= \frac{\tilde{\nu}}{\nu}
		\\
		\mathllap{\tilde{\Omega}} &= \Omega + \frac{\tilde{\nu}}{\kappa^2 d^2} f_{v2}
		&\qquad
		f_{v2} &= 1 - \frac{\chi}{1 + \chi f_{v1}}
		&\qquad
		\Omega &= \sqrt{2 \omega_{ij} \omega_{ij}}
		\\
		\mathllap{\omega_{ij}} &= \frac{1}{2} \left( \pp{u_i}{x_j} - \pp{u_j}{x_i} \right)
		&\qquad
		f_w &= g \left[ \frac{1 + c_{w3}^6}{g^6 + c_{w3}^6} \right]^{1/6}
		&\qquad
		g &= r + c_{w2}(r^6 - r)
		\\
		&&&&
		r &= \min \left[ \frac{\tilde{\nu}}{\tilde{S} \kappa^2 d^2}, 10 \right]
		\;,
		\label{eq:SA_definitions}
	\end{aligned} \\[0.5cm]
	&\begin{aligned}
		\mathllap{c_{b1}} &= 0.1355
		&\quad
		c_{b2} &= 0.622
		&\quad
		\sigma &= 2/3
		&\quad
		\kappa &= 0.41
		\\
		\mathllap{c_{v1}} &= 7.1
		&\quad
		c_{w2} &= 0.3
		&\quad
		c_{w3} &= 2
		&\quad
		c_{w1} &= \frac{c_{b1}}{\kappa^2} + \frac{1 + c_{b2}}{\sigma}
		\;.
		\label{eq:SA_constants}
	\end{aligned}
\end{align}

Because this model has been extensively summarized in prior work for this course, further exposition will be left to \citet{spalart1992}, and we will proceed with a discussion of the curvature correction.

\subsection{Spalart-Shur Curvature-Correction}

The key concepts underpinning the RC-correction were proposed by \citet{spalart1997}, and are understandably similar to those used in developing the SA model due to Spalart's hand in both formulations.

Developed by \citet{shur2000}, the method of accounting for streamline curvature in the standard SA model is relatively simple. In our PHASTA implementation, we assume a stationary reference frame. Thus $\Omega_m' = 0$ in the referenced paper, and Coriolos terms vanish. With this assumption, the only modification required to the standard SA-noft2 model is to multiply the production term $c_{b1} \tilde{S} \tilde{\nu}$ in \eqref{eq:SA_transport} by a rotation function $f_{r1}$, where
\begin{align}
	&\begin{aligned}
		\mathllap{f_{r1}} = (1 + c_{r1}) \frac{2 r^*}{1 + r^*} \left[ 1 - c_{r3} \arctan(c_{r2} \tilde{r}) \right] - c_{r1}
		\;
		\label{eq:SARC_rotation_function}
	\end{aligned}\\[0.25cm]
	\phantom{f_{r1}}
	&\begin{aligned}
		\mathllap{r^*} &= S / \Omega
		\; &\qquad
		\tilde{r} &= \frac{2 \omega_{ik} S_{jk}}{D^4} \left( \frac{D S_{ij}}{D t} \right)
		\; &\qquad
		S_{ij} &= \frac{1}{2} \left( \pp{u_i}{x_j} + \pp{u_j}{x_i} \right)
		\; \\
		\mathllap{S^2} &= 2 S_{ij} S_{ij}
		\; &\qquad
		\Omega^2 &= 2 \omega_{ij} \omega_{ij}
		\; &\qquad
		D^2 &= (S^2 + \Omega^2) / 2
		\label{eq:SARC_definitions}
	\end{aligned}\\[0.25cm]
	&\begin{aligned}
		\mathllap{c_{r1}} &= 1.0
		\; \quad
		c_{r2} = 12
		\; \quad
		c_{r3} = 1.0
		\;.
		\label{eq:SARC_constants}
	\end{aligned}
\end{align}

%%%%%%%%%%%%%%%%%%%%%%%%%%%%%%%%%%%%%%%%%%%%%%%%%
%%%%%%%%%%%%%%%%%%%%%%%%%%%%%%%%%%%%%%%%%%%%%%%%%
\section{PHASTA Implementation} %%%%%%%%%%%%%%%%%
%%%%%%%%%%%%%%%%%%%%%%%%%%%%%%%%%%%%%%%%%%%%%%%%%
%%%%%%%%%%%%%%%%%%%%%%%%%%%%%%%%%%%%%%%%%%%%%%%%%

The SA-noft2 model has already been implemented in PHASTA, so we must only calculate and pre-multiply the production term by $f_{r1}$. The material derivative of the strain tensor, $DS_{ij}/Dt$, in \eqref{eq:SARC_definitions} makes this process slightly more complicated than initially expected. Here, we describe the process used to compute $f_{r1}$ conceptually and in code.

\subsection{Mathematics}

We can write the proper definition of $DS_{ij}/Dt$ as
\begin{equation}
\frac{DS_{ij}}{Dt}
\equiv
\left( \pp{}{t} + u_k \pp{}{x_k} \right) S_{ij}
=
\left( \pp{}{t} + u_k \pp{}{x_k} \right)
\frac{1}{2} \left( \pp{u_i}{x_j} + \pp{u_j}{x_i} \right)
\;,
\end{equation}
from which it is clear that we need to compute two kinds of terms within the code: the temporal derivatives and spatial gradients of the velocity gradient tensor $\partial u_i / \partial x_j$.

For the time term, we can interchange the order of differentiation due to the continuum assumption,
\begin{equation}
\pp{}{t} \pp{u_i}{x_j}
=
\pp{}{x_j} \pp{u_i}{t}
=
\pp{}{x_j} a_i
\;,
\end{equation}
where $a_i$ is the fluid acceleration, which is already computed during each flow solve. The spatial derivatives of $a_i$ can then be computed using shape function gradients in the manner standard to finite element analysis. That is, (math here from FEM)......

%%%%%%%%%%%%%%%%%%%%%%%%%%%%%%%%%%%%%%%%%%%%%%%%%
%%%%%%%%%%%%%%%%%%%%%%%%%%%%%%%%%%%%%%%%%%%%%%%%%
\section{Model Validation} %%%%%%%%%%%%%%%%%%%%%%
%%%%%%%%%%%%%%%%%%%%%%%%%%%%%%%%%%%%%%%%%%%%%%%%%
%%%%%%%%%%%%%%%%%%%%%%%%%%%%%%%%%%%%%%%%%%%%%%%%%

%%%%%%%%%%%%%%%%%%%%%%%%%%%%%%%%%%%%%%%%%%%%%%%%%
%%%%%%%%%%%%%%%%%%%%%%%%%%%%%%%%%%%%%%%%%%%%%%%%%
\section*{Appendix A: PHASTA Code} %%%%%%%%%%%%%%
%%%%%%%%%%%%%%%%%%%%%%%%%%%%%%%%%%%%%%%%%%%%%%%%%
%%%%%%%%%%%%%%%%%%%%%%%%%%%%%%%%%%%%%%%%%%%%%%%%%

All changes discussed here pertain to the incompressible code, including files common to both when appropriate.

In \li{input_fform.f}, the entry \li{RANS-SARC} is added to represent this turbulence model in \li{input.config}.

\bibliographystyle{plainnat}
\bibliography{sources}

%%
%% DOCUMENT END
%%
\end{document}
